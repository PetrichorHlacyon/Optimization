\ifx\allfiles\undefined
\documentclass[12pt, a4paper,oneside, UTF8]{ctexbook}
\usepackage[dvipsnames]{xcolor}
\usepackage{amsmath}   % 数学公式
\usepackage{amsthm}    % 定理环境
\usepackage{amssymb}   % 更多公式符号
\usepackage{graphicx}  % 插图
\usepackage{mathrsfs}  % 数学字体
\usepackage{enumitem}  % 列表
\usepackage{geometry}  % 页面调整
\usepackage{unicode-math}
\usepackage[colorlinks,linkcolor=blue,anchorcolor=blue,citecolor=blue]{hyperref}
\usepackage{tcolorbox}
\usepackage{subfigure}
\usepackage{utfsym}
\usepackage{bm}
\usepackage{diagbox}
\usepackage{tabularx}
\usepackage[]{algorithm, algorithmicx, algpseudocode}
\usepackage{framed}
\tcbuselibrary{most}
\usepackage[]{bm}
\graphicspath{ {img/},{../img/}, {img/}, {../img/} }  % 配置图形文件检索目录
\linespread{1.5} % 行高

% 页码设置
\geometry{top=25.4mm,bottom=25.4mm,left=20mm,right=20mm,headheight=2.17cm,headsep=4mm,footskip=12mm}

% 设置列表环境的上下间距
\setenumerate[1]{itemsep=5pt,partopsep=0pt,parsep=\parskip,topsep=5pt}
\setitemize[1]{itemsep=5pt,partopsep=0pt,parsep=\parskip,topsep=5pt}
\setdescription{itemsep=5pt,partopsep=0pt,parsep=\parskip,topsep=5pt}

% 定理环境
% ########## 定理环境 start ####################################
% #### 将 config.tex 中的定理环境的对应部分替换为如下内容
% 定义单独编号,其他四个共用一个编号计数 这里只列举了五种,其他可类似定义(未定义的使用原来的也可)
\newtcbtheorem[number within=section]{defn}%
{Def.}{colback=OliveGreen!10,colframe=Green!70,fonttitle=\bfseries}{def}

\newtcbtheorem[number within=section]{lemma}%
{Lemma}{colback=Salmon!20,colframe=Salmon!90!Black,fonttitle=\bfseries}{lem}

% 使用另一个计数器 use counter from=lemma
\newtcbtheorem[use counter from=lemma, number within=section]{them}%
{Theorem}{colback=SeaGreen!10!CornflowerBlue!10,colframe=RoyalPurple!55!Aquamarine!100!,fonttitle=\bfseries}{them}

\newtcbtheorem[use counter from=lemma, number within=section]{recall}%
{Recall}{colback=green!5,colframe=green!35!black,fonttitle=\bfseries}{rel}

\newtcbtheorem[use counter from=lemma, number within=section]{remark}%
{Remark}{colback=Emerald!10,colframe=cyan!40!black,fonttitle=\bfseries}{remark}
% colback=red!5,colframe=red!75!black

% 这个颜色我不喜欢
\newtcbtheorem[number within=section]{Notes}%
{Notes}{colback=red!5,colframe=red!75!black,fonttitle=\bfseries}{notes}

\newtcbtheorem[number within=section]{Properties}%
{Properties}{colback=blue!5,colframe=blue!75!black,fonttitle=\bfseries}{Properties}

\newtcbtheorem[number within=section]{Examples}%
{Examples}{colback=red!5,colframe=red!45!black,fonttitle=\bfseries}{Examples}
% .... 命题 例 注 证明 解 使用之前的就可以(全文都是这种框框就很丑了),也可以按照上述定义 ...
% 两种方式定义中文的 证明 和 解 的环境:
% 缺点:\qedhere 命令将会失效【技术有限,暂时无法解决】
\renewenvironment{proof}{\par\textbf{证明.}\;}{\qed\par}
\newenvironment{solution}{\par{\textbf{解.}}\;}{\qed\par}

% 缺点:\bf 是过时命令,可以用 textb f等替代,但编译会有关于字体的警告,不过不影响使用【技术有限,暂时无法解决】
%\renewcommand{\proofname}{\indent\bf 证明}
%\newenvironment{solution}{\begin{proof}[\indent\bf 解]}{\end{proof}}
% ######### 定理环境 end  #####################################

% ↓↓↓↓↓↓↓↓↓↓↓↓↓↓↓↓↓ 以下是自定义的命令  ↓↓↓↓↓↓↓↓↓↓↓↓↓↓↓↓

% 用于调整表格的高度  使用 \hline\xrowht{25pt}
\newcommand{\xrowht}[2][0]{\addstackgap[.5\dimexpr#2\relax]{\vphantom{#1}}}

% 表格环境内长内容换行
\newcommand{\tabincell}[2]{\begin{tabular}{@{}#1@{}}#2\end{tabular}}

% 使用\linespread{1.5} 之后 cases 环境的行高也会改变,重新定义一个 ca 环境可以自动控制 cases 环境行高
\newenvironment{ca}[1][1]{\linespread{#1} \selectfont \begin{cases}}{\end{cases}}
% 和上面一样
\newenvironment{vx}[1][1]{\linespread{#1} \selectfont \begin{vmatrix}}{\end{vmatrix}}

\def\d{\textup{d}} % 直立体 d 用于微分符号 dx
\def\R{\mathbb{R}} % 实数域
\def\K{\mathcal{K}} % any set representation used in definition
\newcommand{\bs}[1]{\boldsymbol{#1}}    % 加粗,常用于向量
\newcommand{\ora}[1]{\overrightarrow{#1}} % 向量

% 数学 平行 符号
\newcommand{\pll}{\kern 0.56em/\kern -0.8em /\kern 0.56em}

% 用于空行\myspace{1} 表示空一行 填 2 表示空两行  
\newcommand{\myspace}[1]{\par\vspace{#1\baselineskip}}

%%%%cite url link%%%%%%%%%
\usepackage{url}

%%%%%%%%%%%%%%%%%%%

%%%%%%% Custom Command %%%%%%%
\renewcommand{\thefootnote}{\arabic{footnote}}

%%%%%%%%%%%%%%%%%%%
\begin{document}
	% \title{{\Huge{\textbf{Convex Optimization Note}}}}
\author{Zelin Yao}
\date{\today}
\maketitle                   % 在单独的标题页上生成一个标题

\thispagestyle{empty}        % 前言页面不使用页码
\begin{center}
	\Huge\textbf{前言}
\end{center}

if people do not believe that mathematics is simple, 
it is only because they do not realize how complicated life is. ——John von Neumann

\begin{flushright}
	\begin{tabular}{c}
		\today \\ 与其焦虑不如先行动起来!
	\end{tabular}
\end{flushright}

\newpage                      % 新的一页
\pagestyle{plain}             % 设置页眉和页脚的排版方式(plain:页眉是空的,页脚只包含一个居中的页码)
\setcounter{page}{1}          % 重新定义页码从第一页开始
\pagenumbering{Roman}         % 使用大写的罗马数字作为页码
\tableofcontents              % 生成目录

\newpage                      % 以下是正文
\pagestyle{plain}
\setcounter{page}{1}          % 使用阿拉伯数字作为页码
\pagenumbering{arabic}
% \setcounter{chapter}{-1}    % 设置 -1 可作为第零章绪论从第零章开始 % 单独编译时,其实不用编译封面目录之类的,如需要不注释这句即可
	\else
	\fi
	%  ↓↓↓↓↓↓↓↓↓↓↓↓↓↓↓↓↓↓↓↓↓↓↓↓↓↓↓↓ 正文部分
	\chapter{Online Optimization} 
	
	This note is the conclusion of \cite{hazan2023introductiononlineconvexoptimization}.
	
	\section{Unconstrained and Constrained Gradient Method for Ordinary Problems}
		The candidate points can be founded in whole space at unconstrained situation. And whatever the points is, it always lands in set. Different from this, the selected points maybe not in set at constrained situation. This causes tiny difference of update formulations.
		\begin{equation*}
			\begin{split}
				x_{k+1} &= x_k -\eta_k \nabla f(x_k) \quad \quad \text{Unconstrained Situation} \\
				x_{k+1} &= [x_k -\eta_k \nabla f(x_k)]_{\mathcal{K}}^+ \quad \quad \text{Constrained Situation}
			\end{split}
		\end{equation*}
		
		\begin{Properties}{The Convergence Rate of Constrained Gradient Descent}{}
			For constrained minimization of $\gamma$\emph{-well-conditioned functions} and step size $\eta_k=1/\beta$, the algorithm of projected gradient method converges as,
			$$
			h_{k+1} = f(x_{k+1})-f(x^*)\le h_1 e^{-\frac{\gamma k}{4}}
			$$
		\end{Properties}
		
		The previous proof methods are complex and abundant and have to analyses from scratch, so a new reduction analysis is explored by \cite{}. This \textbf{reduction method} will used to derive near-optimal convergence rate for non-strongly convex, or non-smooth and so on.
		
	\section{First-Order Algorithms for Online Optimization}
		The problem setting is like that,
		\begin{equation*}
			\begin{split}
				&\min_x \sum_{t=1}^T f_t(x) \text{ or } \min_{x_i,\forall i\in[T]} \sum_{t=1}^T f_t(x_t) \\
				&\text{s.t. } x\in \R^d \text{ or } x_i\in\R^{d_i},\forall i \in {1,2,3,...,T}
			\end{split}
		\end{equation*}
		\subsection{Online Gradient Descent}
			This method is proposed by Zinkevich. In this note, this method only considers the situation that there only has one optimization variable.
			
			\begin{them}{Regret Bound of Online Gradient Descent with Step Size $\eta_t = \frac{D}{G\sqrt{t}}$}{} \label{thm:Regret Bound of Online Gradient Descent}
				Online gradient descent with step size $\eta_t = \frac{D}{G\sqrt{t}}$ guarantees that,
				$$
				Regret_T = \sum_{t=1}^T f_t(x_t) - \min_{z\in\K} \sum_{t=1}^T f_t(z) \le\frac{3}{2}GD\sqrt{t}
				$$
			\end{them}
		
			\begin{them}{General Lower Bound of Regret in the Worst Case}{}
				Any online convex optimization algorithm incurs $\Omega(DG\sqrt{t})$ regret in the worst case.
			\end{them}
		
			\begin{them}{The Regret Bound for Online Gradient Descent for Strongly Convex Functions}{}
				Online gradient descent with step size $\eta_t=\frac{1}{\alpha t}$ achieves the following guarantee for all $T$.
				$$
				Regret_T \le \frac{G^2}{2\alpha}(1+\log T)
				$$
				The selection of step size will affect the regret bound considerably.
			\end{them}
			A special case of online convex optimization is the stochastic optimization, which is to minimize a function $f(x)$ with constrain $x\in\K$.
			\begin{algorithm}[!ht] \label{alg Stochastic Gradient Descent}
				\caption{Stochastic Gradient Descent}
				\begin{algorithmic}[1]
					\Require $\mathcal{O},\K,\{\eta_t\},x_1\in\K$
					\Ensure $\bar{x}_T=\frac{1}{T}\sum_{t=1}^T x_t$
					
					\For {$t=1,2,...,T$}
						\State Let $\tilde{\nabla}_t= \mathcal{O}(x_t)$; 
						\State Update and Project 
						\begin{equation}
							\begin{split}
								y_{t+1} = x_t-\eta_t \tilde{\nabla}_t \\
								x_{t+1} = \prod_{\K} y_{t+1}
							\end{split}
						\end{equation}
					\EndFor
				\end{algorithmic}
			\end{algorithm}
			In \fbox{Alg.\ref{alg Stochastic Gradient Descent}}, the notation $\mathcal{O}(x_t)$ means that sample a gradient at point $x_t$, which is equal to $\tilde{\nabla}_t$, and $\mathbb{E}\{\tilde{\nabla}_t\}=\nabla f(x_t),\mathbb{E}\{||\tilde{\nabla}_t||^2\}\le G^2$.
			\begin{them}{The Convergence of Stochastic Gradient Descent}{}
				\fbox{Alg.\ref{alg Stochastic Gradient Descent}} with step size $\eta_t=\frac{D}{G\sqrt{t}}$ guarantees,
				$$
				\mathbb{E}\{f(\bar{x}_T)\} \le \min_{z\in\K}f(z) +\frac{3GD}{2\sqrt{T}}
				$$
				
				\textbf{proof}:
					\begin{equation*} 
						\begin{split}
							&\mathbb{E}\{f(\bar{x}_T)\} -f(x^*)  \\
							= &\mathbb{E}\{f(\frac{1}{T}\sum_{t=1}^Tx_t)\} -f(x^*) \\
							\le & \mathbb{E}\{\frac{1}{T}\sum_{t=1}^Tf(x_t)\} -f(x^*) \text{ \textbf{(the convexity of function $f$)}} \\
							=& \frac{1}{T}\mathbb{E}\{\sum_{t=1}^T(f(x_t)) -Tf(x^*) \} \\
							=& \frac{1}{T}\mathbb{E}\{\sum_{t=1}^T(f(x_t) -f(x^*)) \} \\
							\le & \frac{1}{T}\mathbb{E}\{\sum_{t=1}^T(\nabla f(x_t)^T(x_t-x^*)) \} \text{\quad($f(x^*)\le f(x_t)+\nabla f(x_t)^T(x^*-x_t)$)} \\
							=& \frac{1}{T}\mathbb{E}\{\sum_{t=1}^T(\tilde{\nabla}_t^T(x_t-x^*)) \} \fbox{Frame.\ref{CoSGD 1}}\\
							=& \frac{1}{T}\mathbb{E}\{\sum_{t=1}^T(f_t(x_t)-f_t(x^*)) \} \fbox{Frame.\ref{CoSGD 2}} \\
							=& \frac{\mathbb{E}\{Regret_T(f_t(x_t))\}}{T} \\
							\le&\frac{3}{2}GD\sqrt{T} \text{\quad (the $G,D$ is belong to the function $f_t(z) = \tilde{\nabla}_t^Tz$)}
						\end{split}
					\end{equation*}
			\end{them}
			"That is the magic of SGD; we have matched the nearly optimal convergence rate of first order methods using extreme cheap iterations." \cite{hazan2023introductiononlineconvexoptimization}
			using extremely cheap iterations. 
			\begin{framed} \label{CoSGD 1}
				\begin{equation} 
					\begin{split}
				    	&\frac{1}{T}\mathbb{E}\{\sum_{t=1}^T(\nabla f(x_t)^T(x_t-x^*)) \} \text{$x_t\in\R^d$}\\
				    	=& \frac{1}{T}\mathbb{E}\{\sum_{t=1}^T \sum_{j=1}^d [\nabla f(x_t)]_j[(x_t-x^*)]_j\}  \\
				    	=&\frac{1}{T}\sum_{t=1}^T \sum_{j=1}^d \mathbb{E}\{[\nabla f(x_t)]_j[(x_t-x^*)]_j\} \\
				    	=&\frac{1}{T}\sum_{t=1}^T \sum_{j=1}^d \mathbb{E}\{[\tilde{\nabla}_t]_j[(x_t-x^*)]_j\} \\
				    	=&\frac{1}{T}\mathbb{E}\{\sum_{t=1}^T(\tilde{\nabla}_t^T(x_t-x^*)) \} 
				    \end{split}
				\end{equation}
			\end{framed}
			\begin{framed} \label{CoSGD 2} 
				Firstly define $f_t(z) = \tilde{\nabla}_t^Tz$. The regret of it is,
				$$
				Regret_T(f_t(z)) = \sum_{t=1}^Tf_t(z) - \min_{x^*\in\K}\sum_{t=1}^Tf_t(x^*)$$
				So, $Regret_T(f_t(x_t)) = \sum_{t=1}^T(f_t(x_t)-f_t(x^*))$.
				And according to \fbox{Thm.\ref{thm:Regret Bound of Online Gradient Descent}},
				$$
				Regret_T(f_t(x_t)) \le\frac{3}{2}GD\sqrt{T}
				$$
				Consider its expectation,
				$$
				\mathbb{E}\{Regret_T(f_t(x_t)) = \sum_{t=1}^T \mathbb{E}\{f_t(x_t)\} - \sum_{t=1}^T\mathbb{E}\{f_t(x^*)\} =  \sum_{t=1}^T \nabla f(x_t)^T(x_t-x^*)
				$$
			\end{framed}
		
		
		
	\section{Second-Order Methods for Online Optimization}
	
	
	\section{Regularization}
	
	\section{Bandit Convex Optimization}
	The only difference between \emph{Bandit Convex Optimization(BCO)} and \emph{Online Convex Optimization(OCO)} is that \emph{BCO} is value oracle but \emph{OCO} is gradient and value oracle. \textbf{Bandit algorithm is a kind of policy to balance exploration and exploitation}\cite{anshitou2021banditzhishifenxiangyuzongjie}. For example, a citizen bought a Lottery and won 10 dollars (the reward is 10 dollars), and next time if he sticks to buy this, the exploration is not enough. Because a better lottery may appears. 
		\begin{remark}{What is "oracle"?}{}
			Assume a network with some nodes, these nodes maintains some local functions. Different with OCO, these local functions in BCO setting likes a machine, which will return value rather than gradient. Such situations are normal, for trying selling items, we only know the prices each time, and do not know the gradient because the price is not a definite function. There are so many oracle models, such as,
			\begin{enumerate}
				\item Value oracle. Given $x$, and return $f(x)$.
				\item Gradient oracle. Given $x$, and return $\nabla f(x)$.
				\item First-order oracle. Given $x$, and return $f(x),\nabla f(x)$.
				\item Stochastic oracle. Given $x$, and return $\nabla f(\tilde x),\mathbb\{\nabla f(\tilde x)\}=\mathbb{E}\{\nabla f(x)\}$.
			\end{enumerate}
		\end{remark}
	
	%  ↑↑↑↑↑↑↑↑↑↑↑↑↑↑↑↑↑↑↑↑↑↑↑↑↑↑↑↑ 正文部分
	\ifx\allfiles\undefined
\end{document}
\fi